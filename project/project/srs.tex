\chapter{Software Requirements Specification}
\section{ Arduino IDE}
\subsection{Basic Information Of The Arduino IDE}
\paragraph{} Introduction to Arduino IDE. Arduino IDE is an open source software that is mainly used for writing and compiling the code into the Arduino Module. It is an official Arduino software, making code compilation too easy that even a common person with no prior technical knowledge can get their feet wet with the learning process.Arduino is an open-source electronics platform based on easy-to-use hardware and software. It's intended for anyone making interactive projects. Learn more about Arduino. Arduino Board.

\large{\paragraph{}The Arduino Integrated Development Environment (IDE) is a cross-platform application (for Windows, macOS, Linux) that is written in functions from C and C++. It is used to write and upload programs to Arduino compatible boards, but also, with the help of third-party cores, other vendor development boards.The source code for the IDE is released under the GNU General Public License, version 2. The Arduino IDE supports the languages C and C++ using special rules of code structuring. The Arduino IDE supplies a software library from the Wiring project, which provides many common input and output procedures. User-written code only requires two basic functions, for starting the sketch and the main program loop, that are compiled and linked with a program stub main() into an executable cyclic executive program with the GNU toolchain, also included with the IDE distribution. The Arduino IDE employs the program avrdude to convert the executable code into a text file in hexadecimal encoding that is loaded into the Arduino board by a loader program in the board's firmware. By default, avrdude is used as the uploading tool to flash the user code onto official Arduino boards.}